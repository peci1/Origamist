\selectlanguage{english}

\chapter*{Introduction}
\addcontentsline{toc}{chapter}{Introduction}

Everyone who has ever tried to fold an origami model knows how important it is to have a \emph{good manual} describing the steps of the folding process. A good manual consists of lots of images (generally one image per step) and their descriptions. 

Why are the manuals needed? It is too hard (if not impossible) for most people to guess the folding process by only seeing the result shape. And this is even harder if you only see the result as a 2-dimensional image on screen or paper. Thus, several methods to aid people with the folding process were introduced.

The most practical help method is learning from someone who knows how to fold the desired model. This method has two disadvantages - a man can forget what he has learnt, and, in most cases, there is simply nobody who knows how to fold the desired model (except in origami communities). Therefore \emph{paper manuals} were invented. They are (conceptually) eternal and are relatively easy to obtain (in books or on the Web). They have another disadvantage - if some steps are unclear in the manual (which happens not so infrequently), there is no other help. The last (relatively new) means to learn the folding process are \emph{video tutorials}.

In the paper manuals (or bitmap image manuals, we will call both of them `paper' manuals) there are some established graphical marks that indicate the operations to be done with the paper in the step. We will discuss these marks in higher detail further in the text. The marks cover the most of operations one would like to do with the paper, but their meanings aren't fixed and unambiguous, which can lead to lack of clarity of the manual.

`Paper' and video manuals share one more disadvantage, too. They aren't \emph{simply editable} by other people. The `paper' manuals are either printed or distributed as PDF or image files, and these aren't simply editable (PDF editors exist, but aren't widely used; editing an image manual involves some non-basic knowledge of computer usage). Video editing is even more difficult. So, ways to edit these manuals exists, but none of them is straightforward.

Origamist brings a new alternative to those types of manuals. It presents the concept of `live' manual. Each folding step is represented as a 3-dimensional model, which the user can view from different viewing angles and zoom levels. Furthermore, everyone can simply edit the model in the Origamist editor. It doesn't matter if the user just wants to add a translation of the steps' descriptions, edit existing step descriptions or if he wants to add some more steps, all of these activities can be done straightforward in the editor.

Also, all of the previously mentioned types of origami manuals (except personal assistance) can be exported from the Origamist application. Only the exported animation has no sound track, which is an important part of video tutorials (but it is possible to add this functionality, also the data model can be simply modified to store this type of descriptions).

%TODO add a short description of each chapter
