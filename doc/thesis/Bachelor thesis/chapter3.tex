\chapter{Origamist basic facts \& algorithms}

Here I will present the basic structure of Origamist, the technologies used, and the algorithms working inside the application.

\section{Basic structure of Origamist}

\subsection{Two parts of the application}
Origamist isn't just a single application. It consists of two standalone parts. 

\emph{Origamist editor} provides the tools needed to create new origami diagrams and manuals. So various folds can be added to existing models, or a brand new model can be created. Also the metadata of the model can be edited.

\emph{Origamist viewer} is a viewer application, that can only display formerly created diagrams. Its main target is to be used as webpage applet, so that the visitors of an origami site can comfortably browse the models the site offers. Although the viewer isn't primarily meant to create the manuals, it has the same export possibilities as the editor, so every user of any part of Origamist is able to create the manuals in various export formats.
